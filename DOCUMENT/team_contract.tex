\documentclass{article}
\usepackage[utf8]{vietnam}
\usepackage{tabularx}
\usepackage{graphicx}
\usepackage{hyperref}  % Add this package for \href support

\title{HỢP ĐỒNG THÀNH LẬP NHÓM}
\date{}

\begin{document}

\maketitle

% Add table of contents
\tableofcontents
\newpage

\section{Thông tin nhóm}
\begin{itemize}
    \item \textbf{Ngày thành lập:} Thứ 4, ngày 23 tháng 10 năm 2024
    \item \textbf{Tên nhóm:} HHNTQ
\end{itemize}

\subsection{Thành viên}
\begin{tabularx}{\textwidth}{|X|X|}
\hline
\textbf{Tên sinh viên} & \textbf{Mã số sinh viên} \\
\hline
Nguyễn Hà Nguyên & 24730050 \\
Nguyễn Chánh Huy & 24730033 \\
Ngô Thị Như Huỳnh & 24730036 \\
Nguyễn Đinh Toàn & 24730072 \\
Trần Xuân Quang & 24730061 \\
\hline
\end{tabularx}

\subsection{Vai trò của các thành viên}
\begin{itemize}
    \item \textbf{Nguyễn Hà Nguyên:} Code các phần trong game. Tạo và cập nhật hợp đồng nhóm. Trưởng nhóm, tổ chức công việc trong nhóm.
    \item \textbf{Nguyễn Đinh Toàn:} Code các phần trong game. Review code. Đảm bảo chất lượng sản phẩm trò chơi.
    \item \textbf{Trần Xuân Quang:} Code các phần trong game. Phụ trách Tài liệu kỹ thuật của trò chơi.
    \item \textbf{Ngô Thị Như Huỳnh:} Code các phần trong game. Phụ trách Mô tả quá trình làm việc nhóm.
    \item \textbf{Nguyễn Chánh Huy:} Code các phần trong game. Phụ trách Viết ra các kỹ năng mà nhóm đã áp dụng khi tham gia đồ án này và Ghi chú phần Đánh giá việc thực hiện hợp đồng nhóm (dưới sự đồng thuận của các thành viên trong nhóm).
\end{itemize}

\section{Mục tiêu}
\begin{itemize}
    \item \textbf{Nắm vững kiến thức:} Hiểu về cách vận hành nhóm hiệu quả.
    \item \textbf{Vận dụng kiến thức:} Áp dụng kiến thức về thực hành nhóm vào việc tạo nên một trò chơi hoàn chỉnh bằng C++ (Snake game).
    \item \textbf{Phát triển kỹ năng:} Nâng cao khả năng làm việc nhóm, hợp tác hiệu quả giữa các thành viên.
    \item \textbf{Đạt được kết quả:} Hoàn thành đồ án cuối kỳ môn Kỹ năng nghề nghiệp (Snake game) và đạt được điểm số cao để qua môn học.
\end{itemize}

\section{Nguyên tắc}
\begin{itemize}
    \item Nhóm hoạt động theo mô hình chặt chẽ.
    \item Nhóm ra quyết định dựa trên ý kiến của trưởng nhóm (tham vấn các thành viên trước khi ra quyết định).
    \item Sẵn sàng tiếp thu những đóng góp và tôn trọng sự đa dạng trong quan điểm.
    \item Giải quyết vấn đề trong nhóm bằng cách chia sẻ thẳng thắn và lắng nghe tích cực.
\end{itemize}

\section{Mô hình hoạt động và công cụ}

\subsection{Mô hình}
\begin{itemize}
    \item Nhóm hoạt động online thông qua các công cụ bên dưới.
    \item Nhóm sẽ liên hệ làm việc qua Email, Slack, Microsoft Teams, Github, v.v.
    \item Mỗi ngày, nhóm sẽ dành ra 5 phút để cập nhật tình hình công việc thông qua tin nhắn Slack bằng cách trả lời 3 câu hỏi: 
    \begin{itemize}
        \item “Hôm qua đã làm gì?”
        \item “Có gặp vấn đề / khó khăn gì không?”
        \item “Hôm nay làm gì?”
    \end{itemize}
\end{itemize}

\subsection{Công cụ}
\begin{itemize}
    \item Trello: \href{https://trello.com/b/8nd8lHck/ss00410-d}{https://trello.com/b/8nd8lHck/ss00410-d}
    \item Github: \href{https://github.com/CN1-2024-1/uit-snake-game}{https://github.com/CN1-2024-1/uit-snake-game}
    \item Overleaf: \href{https://www.overleaf.com/read/gkpxdhndzzfs#c0209d}{https://www.overleaf.com/read/gkpxdhndzzfs#c0209d}
\end{itemize}

\begin{tabularx}{\textwidth}{|X|X|X|}
\hline
\textbf{Chức năng} & \textbf{Tên công cụ} & \textbf{Mô tả chi tiết} \\
\hline
Giao tiếp & Slack (nhóm kín) & Trao đổi công việc chuyên môn theo từng chủ đề, mỗi chủ đề công việc sẽ tạo thành một thread trên Slack. \\
\hline
Quản lý công việc & Trello & Quản lý tình hình của các công việc gồm 4 trạng thái: Defined, Doing, Review, Completed. Mỗi công việc lớn là một danh sách các việc nhỏ. \\
\hline
Quản lý code, sản phẩm & Github & Chia thành 2 loại branch chính: master, dev. Master branch chỉ có thể update bằng cách merge từ một branch khác sau khi được approve. \\
\hline
Tạo code, sản phẩm & Visual Studio Code & Công cụ để viết code và bao gồm các plugin hỗ trợ cho công việc. \\
\hline
Tạo văn bản chuyên môn & Visual Studio Code, Overleaf & Sử dụng Overleaf để các thành viên cùng đóng góp bài làm. Dùng Visual Studio Code để viết văn bản dưới dạng markdown và push lên Github. \\
\hline
\end{tabularx}

\section{Chỉ tiêu đánh giá}
\subsection{Đánh giá}
\begin{tabularx}{\textwidth}{|X|X|X|X|X|}
\hline
\textbf{Tiêu chí} & \textbf{Rất tốt} & \textbf{Tốt} & \textbf{Tạm được} & \textbf{Kém} \\
\hline
Thái độ làm việc & Luôn chủ động, nhiệt tình, tích cực tham gia hoạt động. & Chủ động trong công việc, có tinh thần trách nhiệm. & Thực hiện công việc theo yêu cầu, có tinh thần hợp tác. & Thiếu chủ động, ít tham gia vào hoạt động chung. \\
\hline
Quản lý thời gian & Lên kế hoạch hiệu quả, hoàn thành đúng hạn. & Lên kế hoạch rõ ràng, hoàn thành đúng hạn. & Khó khăn trong việc lên kế hoạch và quản lý thời gian. & Không có kế hoạch làm việc rõ ràng. \\
\hline
Tinh thần trách nhiệm & Luôn hoàn thành công việc, đảm bảo chất lượng. & Hoàn thành công việc đúng hạn, đảm bảo chất lượng. & Cần nhắc nhở để hoàn thành công việc. & Không hoàn thành công việc, đổ lỗi cho người khác. \\
\hline
Giải quyết vấn đề phát sinh & Đưa ra giải pháp sáng tạo, hiệu quả. & Đưa ra giải pháp khả thi. & Khó khăn trong việc tìm ra giải pháp. & Tránh né vấn đề, không đưa ra giải pháp. \\
\hline
Giao tiếp nhóm và trao đổi thông tin & Luôn chia sẻ thông tin rõ ràng, súc tích. & Hợp tác tốt với các thành viên khác. & Tham gia vào thảo luận nhưng chưa thật sự tích cực. & Gần như không giao tiếp, không chia sẻ thông tin. \\
\hline
\end{tabularx}

\section{Đánh giá các thành viên trong nhóm} (Được thực hiện sau khi hoàn thành Snake game)

\subsection{Đánh giá Nguyễn Hà Nguyên}
\begin{tabularx}{\textwidth}{|X|X|X|X|X|}
\hline
\textbf{Đặc điểm} & \textbf{Rất tốt} & \textbf{Tốt} & \textbf{Tạm được} & \textbf{Kém} \\
\hline
Thái độ làm việc &  & & & \\
\hline
Quản lý thời gian & &  & & \\
\hline
Tinh thần trách nhiệm &  & & & \\
\hline
Giải quyết vấn đề phát sinh & &  & & \\
\hline
Giao tiếp nhóm và trao đổi thông tin &  & & & \\
\hline
\end{tabularx}

\subsection{Đánh giá Nguyễn Chánh Huy}
\begin{tabularx}{\textwidth}{|X|X|X|X|X|}
\hline
\textbf{Đặc điểm} & \textbf{Rất tốt} & \textbf{Tốt} & \textbf{Tạm được} & \textbf{Kém} \\
\hline
Thái độ làm việc & &  & & \\
\hline
Quản lý thời gian &  & & & \\
\hline
Tinh thần trách nhiệm & &  & & \\
\hline
Giải quyết vấn đề phát sinh &  & & & \\
\hline
Giao tiếp nhóm và trao đổi thông tin & &  & & \\
\hline
\end{tabularx}

\subsection{Đánh giá Ngô Thị Như Huỳnh}
\begin{tabularx}{\textwidth}{|X|X|X|X|X|}
\hline
\textbf{Đặc điểm} & \textbf{Rất tốt} & \textbf{Tốt} & \textbf{Tạm được} & \textbf{Kém} \\
\hline
Thái độ làm việc &  & & & \\
\hline
Quản lý thời gian & &  & & \\
\hline
Tinh thần trách nhiệm & &  & & \\
\hline
Giải quyết vấn đề phát sinh &  & & & \\
\hline
Giao tiếp nhóm và trao đổi thông tin & &  & & \\
\hline
\end{tabularx}

\subsection{Đánh giá Nguyễn Đinh Toàn}
\begin{tabularx}{\textwidth}{|X|X|X|X|X|}
\hline
\textbf{Đặc điểm} & \textbf{Rất tốt} & \textbf{Tốt} & \textbf{Tạm được} & \textbf{Kém} \\
\hline
Thái độ làm việc &  & & & \\
\hline
Quản lý thời gian & &  & & \\
\hline
Tinh thần trách nhiệm &  & & & \\
\hline
Giải quyết vấn đề phát sinh & &  & & \\
\hline
Giao tiếp nhóm và trao đổi thông tin &  & & & \\
\hline
\end{tabularx}

\subsection{Đánh giá Trần Xuân Quang}
\begin{tabularx}{\textwidth}{|X|X|X|X|X|}
\hline
\textbf{Đặc điểm} & \textbf{Rất tốt} & \textbf{Tốt} & \textbf{Tạm được} & \textbf{Kém} \\
\hline
Thái độ làm việc &  & & & \\
\hline
Quản lý thời gian & &  & & \\
\hline
Tinh thần trách nhiệm &  & & & \\
\hline
Giải quyết vấn đề phát sinh & &  & & \\
\hline
Giao tiếp nhóm và trao đổi thông tin &  & & & \\
\hline
\end{tabularx}

\subsection{Cam kết}
\begin{figure}[htbp]
    \centering
    \subfigure[Nguyễn Hà Nguyên]{\includegraphics[width=0.15\textwidth]{signature/image1.jpg}} \\
    \subfigure[Nguyễn Chánh Huy]{\includegraphics[width=0.2\textwidth]{signature/image2.jpg}} \\
    \subfigure[Ngô Thị Như Huỳnh]{\includegraphics[width=0.2\textwidth]{signature/image3.jpg}}\\
    \subfigure[Nguyễn Đinh Toàn]{\includegraphics[width=0.2\textwidth]{signature/image4.jpg}}\\
    \subfigure[Trần Xuân Quang]{\includegraphics[width=0.2\textwidth]{signature/image5.jpg}}\\
\end{figure}

\end{document}